
%%%%%%%%%%%%%%%%%%%%%%%%%%%%%%%%%%%%%%%%%
% University Assignment Title Page 
% LaTeX Template
% Version 1.0 (27/12/12)
%
% This template has been downloaded from:
% http://www.LaTeXTemplates.com
%
% Original author:
% WikiBooks (http://en.wikibooks.org/wiki/LaTeX/Title_Creation)
%
% License:
% CC BY-NC-SA 3.0 (http://creativecommons.org/licenses/by-nc-sa/3.0/)
% 
% Instructions for using this template:
% This title page is capable of being compiled as is. This is not useful for 
% including it in another document. To do this, you have two options: 
%
% 1) Copy/paste everything between \begin{document} and \end{document} 
% starting at \begin{titlepage} and paste this into another LaTeX file where you 
% want your title page.
% OR
% 2) Remove everything outside the \begin{titlepage} and \end{titlepage} and 
% move this file to the same directory as the LaTeX file you wish to add it to. 
% Then add \input{./title_page_1.tex} to your LaTeX file where you want your
% title page.
%
%%%%%%%%%%%%%%%%%%%%%%%%%%%%%%%%%%%%%%%%%
%\title{Title page with logo}
%----------------------------------------------------------------------------------------
%	PACKAGES AND OTHER DOCUMENT CONFIGURATIONS
%----------------------------------------------------------------------------------------

\documentclass[12pt]{article}
\usepackage[english]{babel}
\usepackage[utf8x]{inputenc}
\usepackage{amsmath}
\usepackage{graphicx}
\usepackage[colorinlistoftodos]{todonotes}

% My Packpage 
\usepackage{glossaries}
\usepackage{bm}
\usepackage{mathtools}
\newcommand\numberthis{\addtocounter{equation}{1}\tag{\theequation}}
\newglossaryentry{latex}
{
    name=latex,
    description={Is a mark up language specially suited 
    for scientific documents}
}
\newacronym{gcd}{GCD}{Greatest Common Divisor}
\makeglossaries
\begin{document}
\begin{titlepage}
\newcommand{\HRule}{\rule{\linewidth}{0.5mm}} % Defines a new command for the horizontal lines, change thickness here
\center % Center everything on the page
\textsc{\LARGE The University of California, Los Angeles}\\[1.5cm] % Name of your university/college
\textsc{\Large Robotics Design Capstone}\\[0.5cm] % Major heading such as course name
\textsc{\large EE 183DB }\\[0.5cm] % Minor heading such as course title
\HRule \\[0.4cm]
{ \huge \bfseries Off-center spinning mass controller for Quadcopters}\\[0.4cm] % Title of your document
\HRule \\[1.5cm]
\begin{minipage}{0.4\textwidth}
\begin{flushleft} \large
\emph{Author:}\\
Lin \textsc{Li} % Your name
\\
Angel \textsc{Jimenez} % Your name
\\
Wilson \textsc{Chang} % Your name
\\
Amirali \textsc{Omidfar} % Your name
\end{flushleft}
\end{minipage}
~
\begin{minipage}{0.4\textwidth}
\begin{flushright} \large
\emph{Professor:} \\
Ankur \textsc{Metha} % Supervisor's Name
\end{flushright}
\end{minipage}\\[1cm]
{\large \today}\\[1cm] % Date, change the \today to a set date if you want to be precise
\includegraphics[scale=0.2]{UCLA_Logo.png}\\[1cm] % Include a department/university logo - this will require the graphicx package
%----------------------------------------------------------------------------------------
\vfill % Fill the rest of the page with whitespace
\end{titlepage}
\begin{abstract}
Your abstract.
\end{abstract}
\section{Symbols}
Here is a list of all symbols used in this paper:
\\
\begin{tabular}{c p{1\textwidth}}
  $\bm{p} = \begin{bmatrix}x \\ y \\ z \end{bmatrix}$ & linear position vectors \\
  $\bm{q} = \begin{bmatrix} q_r \\ q_i \\ q_j \\ q_k \end{bmatrix}$ & angular orientation in quaternion \\
  $\bm{F_{T}}$ & thrust force \\
  $\bm{F_{G}}$ & gravitational force \\  
  $\bm{F_{AB}}$ & reaction force acted from A on B \\
  $\bm{\tau_{AB}}$ & reaction torque acted from A on B \\
  $\bm{\tau_{M}}$ & torque generated by the motor \\
  $\bm{\tau_{RF}}$ & torque generated by the reaction force \\  
  $m_A$ & mass of A \\
  $I_A$ & moment of inertial of A \\
  $S_x, C_x, T_x$ & $\sin(x), \cos(x), \tan(x)$ respectively \\
\end{tabular}\\
% \\ \\ \\ \\ \\ \\
% \newpage
\section{Mathematical Derivation}
\subsection{Appendix}
The Quaternion-derived Rotation matrix is defined as follow,
\begin{align*}
  \prescript{O}{B}{R} = R(\bm{q_B}) =
  \begin{bmatrix}
    q_r^2 + q_i^2 - q_j^2 - q_k^2 & 2q_iq_j - 2q_rq_k & 2q_iq_k + 2q_rq_j \\
    2q_iq_j + 2q_rq_k & q_r^2 - q_i^2 + q_j^2 - q_k^2 & 2q_jq_k - 2q_rq_i \\
    2q_iq_k - 2q_rq_j & 2q_jq_k + 2q_rq_i & q_r^2 - q_i^2 - q_j^2 +q_k^2
  \end{bmatrix}
\end{align*}

\subsection{Quadcopter Body Dynamics}
Forces and Torques:
\begin{align*}
  \prescript{B}{}{\bm{F_{T}}} &=
  \begin{bmatrix}
    0 \\ 0 \\ F_{TB}
  \end{bmatrix} \\
  \prescript{O}{}{\bm{F_{GB}}} &=
  \begin{bmatrix}
    0 \\ 0 \\ -m_b g
  \end{bmatrix} \\
  \prescript{O}{}{\bm{F_{CB}}} &=
  \begin{bmatrix}
    F_{CBx} \\ F_{CBy} \\ F_{CBz} 
  \end{bmatrix} \\
  \prescript{B}{}{\bm{\tau_{CB}}} &=
  \begin{bmatrix}
    \tau_{CBx} \\ \tau_{CBy} \\ -\tau_{M} 
  \end{bmatrix}
\end{align*}

Net Force and Torque
\begin{align}
  \prescript{O}{}{\bm{F_{B, net}}} &= \prescript{O}{}{\bm{F_{GB}}} + \prescript{O}{}{\bm{F_{T}}} + \prescript{O}{}{\bm{F_{CB}}} = m_B \prescript{O}{}{\bm{a_B}} \\
  \prescript{O}{}{\bm{\tau_{B, net}}} &= R(\bm{q_B})\prescript{B}{}{\bm{\tau_{CB}}} = \prescript{O}{}{I_B} \prescript{O}{}{\bm{\alpha_B}}
\end{align}

\subsection{Controller Dynamics}
Forces and Torques:
\begin{align*}
  \prescript{O}{}{\bm{F_{BC}}} &=
  \begin{bmatrix}
    F_{BCx} \\ F_{BCy} \\ F_{BCz}
  \end{bmatrix} \\
  \prescript{O}{}{\bm{F_{GC}}} &=
  \begin{bmatrix}
    0 \\ 0 \\ -m_c g
  \end{bmatrix} \\
  \prescript{C}{}{\bm{\tau_{BC}}} &=
  \begin{bmatrix}
    \tau_{BCx} \\ \tau_{BCy} \\ \tau_{M} 
  \end{bmatrix} \\
  \prescript{O}{}{\bm{r_{CB}}} &= R(\bm{q_C}) \begin{bmatrix}
                                   -L_{Mx} \\ 0 \\ -L_{Mz}
                                 \end{bmatrix} \\
  \prescript{O}{}{\bm{\tau_{RF}}} &=  \prescript{O}{}{\bm{r_{CB}}} \times \prescript{O}{}{\bm{F_{BC}}}
\end{align*}
Net Force and Net Torque:
\begin{align}
  \prescript{O}{}{\bm{F_{net, C}}} &= \prescript{O}{}{\bm{F_{BC}}} + \prescript{O}{}{\bm{F_{GC}}} = m_C \prescript{O}{}{\bm{a_C}} \\
  \prescript{O}{}{\bm{\tau_{net, C}}} &= R(\bm{q_C})\prescript{C}{}{\bm{\tau_{BC}}} + \prescript{O}{}{\bm{\tau_{RF}}} = \prescript{O}{}{I_c} \prescript{O}{}{\bm{\alpha_C}}
\end{align}
\subsection{Constraints and Manipulation}
The two bodies are contrainted (attached together), there are some relationship between the states and the forces between the body and the controller, 
\begin{align}
  \intertext{Let $\bm{p_{sys}} = \bm{p_B}$ and $\bm{q_{sys}} = \bm{q_{B}}$,}
  \begin{bmatrix}
    \bm{p_C} \\
    \bm{q_C}
  \end{bmatrix} &=
  \begin{bmatrix}
    \bm{p_B} + \bm{r_{BC}} \\
    \bm{q_{\theta}}\bm{q_B}
  \end{bmatrix}
  =
  \begin{bmatrix}
    \bm{p_{sys}} + \bm{r_{BC}} \\
    \bm{q_{\theta}}\bm{q_{sys}}
  \end{bmatrix}  
  \intertext{Newton's Third Law}
  \prescript{O}{}{\bm{F_{BC}}} &= -\prescript{O}{}{\bm{F_{CB}}} \\
  \prescript{O}{}{\bm{\tau_{BC}}} &= -\prescript{O}{}{\bm{\tau_{CB}}}
\end{align}
Combining the above equations (1) - (7), we would like to do some algebraic manipulation to get rid of the unwanted parameters in our state evolution equations. \par
\subsubsection{Combining the Force equations}
\begin{align*}
  \shortintertext{From (1),}
  \prescript{O}{}{\bm{F_{CB}}} &= m_B \prescript{O}{}{\bm{a_B}} - \prescript{O}{}{\bm{F_{GB}}} - \prescript{O}{}{\bm{F_{T}}} \\
  \shortintertext{From (3),}
  \prescript{O}{}{\bm{F_{BC}}} &= m_C \prescript{O}{}{\bm{a_C}} - \prescript{O}{}{\bm{F_{GC}}} \\
  \shortintertext{Using (6),}
  m_B \prescript{O}{}{\bm{a_B}} + m_C \prescript{O}{}{\bm{a_C}} &= \prescript{O}{}{\bm{F_{GC}}} + \prescript{O}{}{\bm{F_{GB}}} + \prescript{O}{}{\bm{F_{T}}} \\
\end{align*}  
Since $\bm{p_\theta}$ is a constant vector, $\bm{a_C} = \bm{a_B} = \bm{\ddot{p}_{sys}}$, the above equation becomes
  \begin{equation}
  \bm{\ddot{p}_{sys}} = \frac{1}{m_B + m_C} (\prescript{O}{}{\bm{F_{GC}}} + \prescript{O}{}{\bm{F_{GB}}} + \prescript{O}{}{\bm{F_{T}}})
\end{equation}
\subsubsection{Combining the Torqe equations}
\begin{align*}
  \shortintertext{From (2),}
  \prescript{O}{}{\bm{\tau_{CB}}} &= \prescript{O}{}{I_B} \prescript{O}{}{\bm{\alpha_B}} \\
  \shortintertext{From (4),}
  \prescript{O}{}{\bm{\tau_{BC}}} &= \prescript{O}{}{I_c} \prescript{O}{}{\bm{\alpha_C}} - \prescript{O}{}{\bm{\tau_{RF}}}  \\
  \shortintertext{Using (7),}
  \prescript{O}{}{I_B} \prescript{O}{}{\bm{\alpha_B}} + \prescript{O}{}{I_c} \prescript{O}{}{\bm{\alpha_C}} &= \prescript{O}{}{\bm{\tau_{RF}}} \\
\end{align*}  
Assuming all the vectors are represented in the inertial O frame, using the quaternion representation for angular acceleration,
\begin{equation}
  2I_B \left[\bm{\ddot{q}_B}\bm{q_B^*} - (\bm{\dot{q}_B} \bm{q_B^*})^2\right] + 2I_c \left[\bm{\ddot{q}_C}\bm{q_C^*} - (\bm{\dot{q}_C} \bm{q_C^*})^2\right] = \prescript{O}{}{\bm{\tau_{RF}}}
\end{equation}
Since $\bm{q_B} = \bm{q_{sys}}$ and $\bm{q_C} = \bm{q_{\theta}q_{sys}}$, by Chain Rule,
\begin{align*}
  \bm{\dot{q}_C} &= \bm{q_\theta} \bm{\omega_{sys}} + \bm{\omega_\theta}\bm{q_{sys}} \\
  \bm{\ddot{q}_C} &= \bm{q_\theta} \bm{\dot{\omega}_{sys}} + 2(\bm{\omega_\theta \omega_{sys}}) + \bm{\alpha_\theta}\bm{q_{sys}}
\end{align*}
where $\bm{\omega_{sys}} = \bm{\dot{q}_{sys}}$, $\quad\bm{\omega_{\theta}} = \bm{\dot{q}_{\theta}}$, $\quad \bm{\alpha_{\theta}} = \bm{\ddot{q}_{\theta}}$.
Combining the above, (9) becomes,
\begin{align*}
  &2I_B[\bm{\dot{\omega}_{sys}} \bm{q_{sys}^*}] + 2I_C[\bm{q_\theta}\bm{\dot{\omega}_{sys}} (\bm{q_\theta}\bm{q_{sys}})^{\bm{*}}] \\
  &= \prescript{O}{}{\bm{\tau_{RF}}} + 2I_B(\bm{\omega_{sys}}\bm{q_{sys}^*})^2 + 2I_C[(\bm{q_\theta} \bm{\omega_{sys}} + \bm{\omega_\theta}\bm{q_{sys}})(\bm{q_{\theta}\bm{q_{sys}}})^{\bm{*}}]^2 - 2I_C[2(\bm{\omega_\theta \omega_C}) + \bm{\alpha_\theta} \bm{q_C}](\bm{q_\theta q_{sys}})^{\bm{*}} \\
\end{align*}
Let the above R.H.S. sum be $\zeta$, we have,
\begin{equation}
  \bm{\dot{\omega}_{sys}} = 2(I_B + I_CR(\bm{q_\theta}))^{-1} \zeta \bm{q_{sys}}
\end{equation}

\subsubsection{State Evolution Equation}
From (8) and (10), we have the follow state evolution equation,
\begin{align*}
  \bm{s_{sys}} =
  \begin{bmatrix}
    \bm{\dot{p}_{sys}} \\
    \bm{\dot{q}_{sys}} \\
    \bm{\dot{q}_{theta}} \\
    \bm{\dot{\omega}_{theta}} \\    
    \bm{\dot{v}_{sys}} \\
    \bm{\dot{\omega}_{sys}} \\    
  \end{bmatrix} =
  \begin{bmatrix}
    \bm{v_{sys}} \\
    \bm{\omega_{sys}} \\
    \bm{\omega_{theta}} \\
    \bm{\alpha_{thete}} \\
    \frac{1}{m_B + m_C} (\prescript{O}{}{\bm{F_{GC}}} + \prescript{O}{}{\bm{F_{GB}}} + \prescript{O}{}{\bm{F_{T}}}) \\
    2(I_B + I_CR(\bm{q_\theta}))^{-1} \zeta \bm{q_{sys}}    
  \end{bmatrix}
\end{align*}
\end{document}