%%%%%%%%%%%%%%%%%%%%%%%%%%%%%%%%%%%%%%%%%
% University Assignment Title Page 
% LaTeX Template
% Version 1.0 (27/12/12)
%
% This template has been downloaded from:
% http://www.LaTeXTemplates.com
%
% Original author:
% WikiBooks (http://en.wikibooks.org/wiki/LaTeX/Title_Creation)
%
% License:
% CC BY-NC-SA 3.0 (http://creativecommons.org/licenses/by-nc-sa/3.0/)
% 
% Instructions for using this template:
% This title page is capable of being compiled as is. This is not useful for 
% including it in another document. To do this, you have two options: 
%
% 1) Copy/paste everything between \begin{document} and \end{document} 
% starting at \begin{titlepage} and paste this into another LaTeX file where you 
% want your title page.
% OR
% 2) Remove everything outside the \begin{titlepage} and \end{titlepage} and 
% move this file to the same directory as the LaTeX file you wish to add it to. 
% Then add \input{./title_page_1.tex} to your LaTeX file where you want your
% title page.
%
%%%%%%%%%%%%%%%%%%%%%%%%%%%%%%%%%%%%%%%%%
%\title{Title page with logo}
%----------------------------------------------------------------------------------------
%	PACKAGES AND OTHER DOCUMENT CONFIGURATIONS
%----------------------------------------------------------------------------------------

\documentclass[12pt]{article}
\usepackage[english]{babel}
\usepackage[utf8x]{inputenc}
\usepackage{amsmath}
\usepackage{graphicx}
\usepackage[colorinlistoftodos]{todonotes}
\usepackage{bm}
% My Packpage 
\usepackage{glossaries}



\newglossaryentry{latex}
{
    name=latex,
    description={Is a mark up language specially suited 
    for scientific documents}
}
\newacronym{gcd}{GCD}{Greatest Common Divisor}
\makeglossaries
\begin{document}
\begin{titlepage}
\newcommand{\HRule}{\rule{\linewidth}{0.5mm}} % Defines a new command for the horizontal lines, change thickness here
\center % Center everything on the page
\textsc{\LARGE UCLA }\\[1.5cm] % Name of your university/college
\textsc{\Large Robotics Design Capstone}\\[0.5cm] % Major heading such as course name
\textsc{\large 183DB }\\[0.5cm] % Minor heading such as course title
\HRule \\[0.4cm]
{ \huge \bfseries Off center spinning mass controller for Quad Copters}\\[0.4cm] % Title of your document
\HRule \\[1.5cm]
\begin{minipage}{0.4\textwidth}
\begin{flushleft} \large
\emph{Author:}\\
Lin \textsc{Li} % Your name
\\
Angel \textsc{} % Your name
\\
Wilson \textsc{} % Your name
\\
Amirali \textsc{Omidfar} % Your name
\end{flushleft}
\end{minipage}
~
\begin{minipage}{0.4\textwidth}
\begin{flushright} \large
\emph{Professor:} \\
Ankur \textsc{Metha} % Supervisor's Name
\end{flushright}
\end{minipage}\\[2cm]
{\large \today}\\[2cm] % Date, change the \today to a set date if you want to be precise
\includegraphics[scale=0.2]{UCLA_Logo.png}\\[1cm] % Include a department/university logo - this will require the graphicx package
%----------------------------------------------------------------------------------------
\vfill % Fill the rest of the page with whitespace
\end{titlepage}
\begin{abstract}
Your abstract.
\end{abstract}
\section{Symbols}
Here is a list of all symbols used in this paper:
\\
\begin{tabular}{c p{1\textwidth}}
  $\bm{\xi}$ & linear position vectors  \\
  $\bm{\eta}$ & angular position vectors \\
  $\alpha$ & roll angle \\
  $\beta$ & pitch angle \\
  $\gamma$ & yaw angle \\
  $\bm{V_B}$ & linear velocity vectors in Body frame\\
  $\bm{\nu_B}$ & angular velocity vectors in Body frame\\
  $\bm{R}$ & rotation matrix from body to inertial frame\\
  $S_x, C_x, T_x$ & $\sin(x), \cos(x), \tan(x)$ respectively \\
\end{tabular}\\
% \\ \\ \\ \\ \\ \\
% \newpage
\section{Mathematical Derivation}
\subsection{Free Body Diagram}
\subsection{Inertial / Body / Controller frame transformation}
The Rotation matrix is shown below,

\begin{align*}
  \bm{R} =
  \begin{bmatrix}
    C_{\gamma}C_{\beta} & C_{\gamma}S_{\beta}S_{\alpha} - S_{\alpha}C_{\alpha} & C_{\gamma}S_{\beta}C_{\alpha} + S_{\gamma}S_{\alpha} \\
    S_{\gamma}C_{\beta} & S_{\beta}S_{\theta}S_{\alpha} + C_{\gamma]}C_{\alpha} & S_{\gamma}S_{\beta}C_{alpha} - C_{\gamma}S_{\alpha} \\
    -S_{\beta} & C_{\beta}S_{\alpha} & C_{\beta}C_{\alpha}
  \end{bmatrix}
\end{align*}
\subsection{Newton-Euler equations}
\begin{align*}
  m\dot{\bm{V}_B} + \bm{\xi}_B \times (m \bm{V_B}) = \bm{R}^T \bm{G} + \bm{T}_B
\end{align*}
\end{document}