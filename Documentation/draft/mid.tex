
%%%%%%%%%%%%%%%%%%%%%%%%%%%%%%%%%%%%%%%%%
% University Assignment Title Page 
% LaTeX Template
% Version 1.0 (27/12/12)
%
% This template has been downloaded from:
% http://www.LaTeXTemplates.com
%
% Original author:
% WikiBooks (http://en.wikibooks.org/wiki/LaTeX/Title_Creation)
%
% License:
% CC BY-NC-SA 3.0 (http://creativecommons.org/licenses/by-nc-sa/3.0/)
% 
% Instructions for using this template:
% This title page is capable of being compiled as is. This is not useful for 
% including it in another document. To do this, you have two options: 
%
% 1) Copy/paste everything between \begin{document} and \end{document} 
% starting at \begin{titlepage} and paste this into another LaTeX file where you 
% want your title page.
% OR
% 2) Remove everything outside the \begin{titlepage} and \end{titlepage} and 
% move this file to the same directory as the LaTeX file you wish to add it to. 
% Then add \input{./title_page_1.tex} to your LaTeX file where you want your
% title page.
%
%%%%%%%%%%%%%%%%%%%%%%%%%%%%%%%%%%%%%%%%%
%\title{Title page with logo}
%----------------------------------------------------------------------------------------
%	PACKAGES AND OTHER DOCUMENT CONFIGURATIONS
%----------------------------------------------------------------------------------------

\documentclass[12pt]{article}
\usepackage[english]{babel}
\usepackage[utf8x]{inputenc}
\usepackage{amsmath}
\usepackage{graphicx}
\usepackage[colorinlistoftodos]{todonotes}

% My Packpage 
\usepackage{glossaries}
\usepackage{bm}
\usepackage{mathtools}
\newcommand\numberthis{\addtocounter{equation}{1}\tag{\theequation}}
\newglossaryentry{latex}
{
    name=latex,
    description={Is a mark up language specially suited 
    for scientific documents}
}
\newacronym{gcd}{GCD}{Greatest Common Divisor}
\makeglossaries
\begin{document}
\begin{titlepage}
\newcommand{\HRule}{\rule{\linewidth}{0.5mm}} % Defines a new command for the horizontal lines, change thickness here
\center % Center everything on the page
\textsc{\LARGE The University of California, Los Angeles}\\[1.5cm] % Name of your university/college
\textsc{\Large Robotics Design Capstone}\\[0.5cm] % Major heading such as course name
\textsc{\large EE 183DB }\\[0.5cm] % Minor heading such as course title
\HRule \\[0.4cm]
{ \huge \bfseries Off-center spinning mass controller for Quadcopters}\\[0.4cm] % Title of your document
\HRule \\[1.5cm]
\begin{minipage}{0.4\textwidth}
\begin{flushleft} \large
\emph{Author:}\\
Lin \textsc{Li} % Your name
\\
Angel \textsc{Jimenez} % Your name
\\
Wilson \textsc{Chang} % Your name
\\
Amirali \textsc{Omidfar} % Your name
\end{flushleft}
\end{minipage}
~
\begin{minipage}{0.4\textwidth}
\begin{flushright} \large
\emph{Professor:} \\
Ankur \textsc{Metha} % Supervisor's Name
\end{flushright}
\end{minipage}\\[1cm]
{\large \today}\\[1cm] % Date, change the \today to a set date if you want to be precise
\includegraphics[scale=0.2]{UCLA_Logo.png}\\[1cm] % Include a department/university logo - this will require the graphicx package
%----------------------------------------------------------------------------------------
\vfill % Fill the rest of the page with whitespace
\end{titlepage}
\begin{abstract}
Your abstract.
\end{abstract}
\section{Symbols}
Here is a list of all symbols used in this paper:
\\
\begin{tabular}{c p{1\textwidth}}
  $\bm{p} = \begin{bmatrix}x \\ y \\ z \end{bmatrix}$ & linear position vectors \\
  $\bm{q} = \begin{bmatrix} q_r \\ q_i \\ q_j \\ q_k \end{bmatrix}$ & angular orientation in quaternion \\
  $\bm{F_{T}}$ & thrust force \\
  $\bm{F_{G}}$ & gravitational force \\  
  $\bm{F_{AB}}$ & reaction force acted from A on B \\
  $\bm{\tau_{AB}}$ & reaction torque acted from A on B \\
  $\bm{\tau_{M}}$ & torque generated by the motor \\
  $\bm{\tau_{RF}}$ & torque generated by the reaction force \\  
  $m_A$ & mass of the A \\
  $I_A$ & moment of inertial of A \\
  $S_x, C_x, T_x$ & $\sin(x), \cos(x), \tan(x)$ respectively \\
\end{tabular}\\
% \\ \\ \\ \\ \\ \\
% \newpage
\section{Mathematical Derivation}

\subsection{Assumptions}
\begin{itemize}
\item Assume unit quaternions: $||\bm{q}|| = 1$
\end{itemize}
\newpage
\subsection{}
Assuming all the vectors are represented in the inertial O frame, using the quaternion representation for angular acceleration,
\begin{equation}
  2I_B \left[\bm{\ddot{q}_B}\bm{q_B^*} - (\bm{\dot{q}_B} \bm{q_B^*})^2\right] + 2I_c \left[\bm{\ddot{q}_C}\bm{q_C^*} - (\bm{\dot{q}_C} \bm{q_C^*})^2\right] = \prescript{O}{}{\bm{\tau_{RF}}}
\end{equation}
Since $\bm{q_B} = \bm{q_{sys}}$ and $\bm{q_C} = \bm{q_{\theta}q_{sys}}$, by Chain Rule,
\begin{align*}
  \bm{\dot{q}_C} &= \bm{q_\theta} \bm{\omega_{sys}} + \bm{\omega_\theta}\bm{q_{sys}} \\
  \bm{\ddot{q}_C} &= \bm{q_\theta} \bm{\dot{\omega}_{sys}} + 2(\bm{\omega_\theta \omega_{sys}}) + \bm{\alpha_\theta}\bm{q_{sys}}
\end{align*}
where $\bm{\omega_{sys}} = \bm{\dot{q}_{sys}}$, $\quad\bm{\omega_{\theta}} = \bm{\dot{q}_{\theta}}$, $\quad \bm{\alpha_{\theta}} = \bm{\ddot{q}_{\theta}}$.
Combining the above, (9) becomes,
\begin{align*}
  &2I_B[\bm{\dot{\omega}_{sys}} \bm{q_{sys}^*}] + 2I_C[\bm{q_\theta}\bm{\dot{\omega}_{sys}} (\bm{q_\theta}\bm{q_{sys}})^{\bm{*}}] \\
  &= \prescript{O}{}{\bm{\tau_{RF}}} + 2I_B(\bm{\omega_{sys}}\bm{q_{sys}^*})^2 + 2I_C[(\bm{q_\theta} \bm{\omega_{sys}} + \bm{\omega_\theta}\bm{q_{sys}})(\bm{q_{\theta}\bm{q_{sys}}})^{\bm{*}}]^2 - 2I_C[2(\bm{\omega_\theta \omega_C}) + \bm{\alpha_\theta} \bm{q_C}](\bm{q_\theta q_{sys}})^{\bm{*}} \\
\end{align*}
\end{document}